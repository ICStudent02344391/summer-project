\documentclass[11pt]{article}
\usepackage[utf8]{inputenc}
\usepackage[margin=1in]{geometry}
\usepackage{natbib}
\bibliographystyle{abbrvnat}
\usepackage{amsmath}
\usepackage{amssymb}
\usepackage{comment}
\usepackage{bbm}
\usepackage{amsthm}
\geometry{a4paper}
\usepackage{graphicx}
\usepackage{booktabs}
\usepackage{array}
\usepackage{paralist}
\usepackage{verbatim}
\usepackage{subfig}
\usepackage{multirow}
\usepackage{rotating}
\usepackage{fancyhdr}
\usepackage{hyperref}
\usepackage{bm}
\usepackage{xcolor}
\usepackage{listings}
\lstset{basicstyle=\ttfamily,
  showstringspaces=false,
  commentstyle=\color{red},
  keywordstyle=\color{blue}
}
\pagestyle{fancy}
\renewcommand{\headrulewidth}{0pt}
\lhead{}\chead{}\rhead{}
\lfoot{}\cfoot{\thepage}\rfoot{}
\usepackage{algorithm}
\usepackage{algpseudocode}
\usepackage{fancyvrb}
\usepackage{pythonhighlight}

\title{%
  Summer project \\
  \large Learning to bid above a threshold.}
\date{May 2024}
\author{02344391}

\begin{document}

\maketitle

MSc projects 2024 - Ciara Pike-Burke:

"

3. Learning to Bid Above a Threshold

Consider the problem of bidding for a product over a period of $T$ rounds. In each round, if
we bid above the value, we win the item. However, we do not want to bid too far above the
value of the item since that would mean that we overpay for the product. The problem is
that we assume that the threshold of the true value is noisy (so varies slightly from round
to round) and unknown so must be learnt while simultaneously learning to bid a price that
wins but is not too far above the threshold. There are two models we could consider. In the
first, the expected threshold is stationary but we only observe binary feedback on whether
our bid was accepted or not. This could be extended to consider the case that after some
delay, we observe the true value. In the second, the threshold evolves according to some time
series model and we only observe the price if we win. The project would involve coming up
with algorithms for this problem and investigating them in a simulation study. To the best
of our knowledge, there are no prior works studying this exact problem, but some related
works are [\cite{NIPS2016_0bf727e9}, \cite{badanidiyuru2021learning}, \cite{zhang2023learning}].

"

We introduce some useful notations:

\paragraph*{Model 1}

\begin{itemize}
  \item Probability space $(\Omega, \mathcal{F}, \mathbb{P})$ with filtration $\mathbb{F} = (\mathcal{F}_t)_{t=1}^T$;
  \item Threshold $\tau_t$ is stationary: for each $t = 1, \dots T$, $\tau_t = \tau + \epsilon_t$, where $\tau$ is a constant
and $(\epsilon_t)_{t = 1}^T$ i.i.d with mean $0$ and variance $\sigma^2$;
  \item At each round $t$, we propose a bid: $b_t$;
  \item At each round $t$, we have the information: $\delta_t = \mathbb{I}_{\{\tau_t \leq b_t\}}$;
  \item Extension: after some delay $d$, we observe the true value. Thus for $t > d$, $\tau_{t-d}$ is $\mathcal{F}_t$- measurable.
\end{itemize}

The goal is to find a strategy that allows for bids above the threshold as quickly as possible while minimising the difference 
between the threshold and the bid. 
If we consider the model with the true threshold values provided with a delay, this information should be used to model the 
distribution of the thresholds, particularly to find the variance and provide a confidence interval that the bid is above the threshold.

\paragraph*{Model 2}

\begin{itemize}
  \item Probability space $(\Omega, \mathcal{F}, \mathbb{P})$ with filtration $\mathbb{F} = (\mathcal{F}_t)_{t=1}^T$;
  \item Threshold $\tau_t$ evolves according to some time series;
  \item At each round $t$, we propose a bid: $b_t$;
  \item At each round $t$, we have the information: $X_t = \tau_t \mathbb{I}_{\{\tau_t \leq b_t\}}$;
\end{itemize}

One strategy would be to quickly find bid values that provide information about the time series, and then use this information to model the time series 
from the censored data $X_t$. This could be done in episodes, where each episode contains $T_s$ time steps, and the bid selection in each episode is based on 
the assumed distribution of thresholds found in the previous episode. We could initially assume that the distribution follows an ARMA(.,.), GARCH(.,.), 
or similar model, and we aim to find the parameters and update their estimators at each episode.

\newpage
\bibliography{refs}
\end{document}